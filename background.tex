% 10 pts
\section{Background}
\label{sec:background}

Expand on what the reader needs to know and understand to adequately describe
the problem. This may include definitions of terms and explanations of the
present workflow people use to solve the task. Do not assume people are
familiar with your particular research problem. Aim at the knowledge of an
an average undergraduate computer science degree holder.

This section should answer the question ``What does someone unfamiliar with the
area need to know to understand this project?'' Often times this section
focuses on the \textit{domain}. For example, a new visualization for biology
data will explain the type of biology data used, the terminology that will be
used to describe it, the people who use it, and what they hope to learn from
it. If there is no specific domain, the background section may describe
general terms about the data abstraction (e.g. ``We define a graph $G = (V, E)$
as...''), common practices, or other information for visualization researchers
who are not as familiar with this problem in comparison to all the others. 

% 5 pts
\subsection{Related Work}
\label{sec:related}

Discuss the work related to your project---include both visualization and
domain-specific references to the problem you're trying to solve. Prominent
related works should be included for this milestone. If you are proposing a
literature review or task taxonomy, this section must include a discussion of
what similar ones exist already and should tie into the motivation. Is the
existing one old enough to be missing many relevant advances?

For example, if my project involves analyzing perfromance data of distributed
systems, I might want to cite Perfopticon~\cite{Moritz:2015:EuroVis}. If my
project involves designing a new library to support creating new
visualizations, I might want to cite d3js~\cite{d3js}. Note here I am using
keys such as "Moritz:2015:EuroVis" for citations. These refer to the full
definitions in proposal.bib.

ACM Digital Library makes it easy to get bib files for proposal.bib and there
are guides online for citing things like books~\cite{ware:2004:IVP},
theses~\cite{levoy:1989:DSV}, journal articles~\cite{Lorensen:1987:MCA}, and
conference proceedings~\cite{Nielson:1991:TAD}. There's even a format for
miscellaneous references (misc) such as websites and libraries not associated
with a article.

